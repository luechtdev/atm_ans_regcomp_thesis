%*******************************************************
% Abstract in German
%*******************************************************
\begin{otherlanguage}{ngerman}
    \pdfbookmark[1]{Zusammenfassung}{Zusammenfassung}
    \chapter*{Zusammenfassung}

    \graffito{Wird noch ein wenig aktualisiert}

    In dieser Arbeit werden die Möglichkeiten der Nutzung elektronisch verfügbaren Informationen zu rechtlichen und regulativen Nachweisdokumenten und externen regulativen Anforderungen an ATM/ANS Einrichtungen analysiert. Dabei sollen Auswirkungen regulativer Änderungen auf die interne Nachweisführung zu reflektiert werden. 
    Es wird erläutert, welche Anforderungen für die Zertifizierung bzw. für die Konformitätserklärung  zur Inbetriebnahme von ATM/ANS Equipments relevant sind und wie diese im Rahmen des ordentlichen Gesetzgebungsverfahrens erstellt und überarbeitet werden. 
    Hierfür sollen sowohl der Aufbau sowie die Verfügbarkeit der öffentlichen Informationsquellen analysiert und verglichen werden, um zu überprüfen, inwiefern -- und mit welchem Aufwand -- aus dem Lifecycle der regulativen Anforderungen Rückschlüsse auf Änderungen in der Nachweisführung gezogen werden können.
 
\end{otherlanguage}
