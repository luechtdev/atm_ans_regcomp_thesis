%*******************************************************
% Abstract in German
%*******************************************************
\begin{otherlanguage}{ngerman}
    \pdfbookmark[1]{Zusammenfassung}{Zusammenfassung}
    \chapter*{Zusammenfassung}

    In dieser Arbeit werden die Möglichkeiten der Nutzung elektronisch verfügbarer Informationen zu rechtlichen und regulativen Nachweisdokumenten und anderen externen regulativen Anforderungen an ATM/ANS Einrichtungen analysiert. 
    Dabei sollen Auswirkungen regulativer Änderungen auf die interne Nachweisführung reflektiert werden. 
    Es wird im Folgenden erläutert, welche Anforderungen für die Zertifizierung bzw. für die Konformitätserklärung  zur Inbetriebnahme von ATM/ANS Equipment relevant sind und wie diese im Rahmen der gesetzgebenen Organisationen erstellt und überarbeitet werden. 
    Hierfür sollen sowohl der Aufbau sowie die Verfügbarkeit der öffentlichen Informationsquellen analysiert und bewertet werden, um zu überprüfen, inwiefern -- und mit welchem Aufwand -- aus dem Lifecycle der regulativen Anforderungen Rückschlüsse auf Änderungen in der Nachweisführung gezogen werden können.
    Für den Rahmen der Analyse wird dieser Prozess auf eine Impact-Analyse abgebildet und anhand bekannter Prozesse und Metriken bewertet.
 
\end{otherlanguage}
