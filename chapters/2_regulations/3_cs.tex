
% \graffito{Der Text ist noch aus der PP1 Dokumentation, ich bereite den noch auf will das Kapitel aber nicht zu lang halten}

\acf{CS} bilden die spezifischste, dritte Abstraktionsebene des \ac{SES}-Rahmens.
Sie stehen außerhalb des verpflichtenden Teils der Regularien und dessen Anwendung ist für alle teilnehmenden Parteien freiwillig. 
Generell beschreiben 
    
    \begin{itemize}
        \item einen Standard für Systeme oder Komponenten, welcher von einer \ac{ESO} wie \ac{CEN}, \ac{CENELEC}, \ac{ETSI} in Kooperation mit \ac{EUROCAE}, auf einem Mandat der Europäischen Kommission beschlossen wurde, oder
        \item eine Spezifikation bezüglich der operativen Koordination zwischen den \acp{ANSP}, welche auf Ersuchen der Europäischen Kommission von EUROCONTROL beschlossen wurde.
    \end{itemize}

\noindent
Angenommene \ac{CS} werden im \ac{OJ} veröffentlicht.
Beispielsweise:

\begin{itemize}
    \item \ac{ETSI} EN 303 214 V1.2.1\footnote{\ac{OJ} 2012/C/168/03}: 
        \acf{DLS} System: Requirements for ground constituents and system testing
    \item EUROCONTROL Spec-0100 Edition 2.0\footnote{\ac{OJ} 2007/C/188/03}:
        EUROCONTROL Specification of Interoperability and performance requirements for the \acf{FMTP}
\end{itemize}

\noindent
Neben den bereits veröffentlichten \ac{CS} wird weiterhin von den \acp{ESO} an Standards gearbeitet oder diese initiiert. 
Diese Arbeit basiert auf den Standardisierungsmandaten der Europäischen Kommission an die \acp{ESO}, welche die Ziele der zu erstellenden \ac{CS} darlegen.
Diese umfassen die Mandate\footnote{\href{https://portal.etsi.org/EC-EFTA-Mandates}{https://portal.etsi.org/EC-EFTA-Mandates}}:
\begin{itemize}
    \item M/390: \acf{SWAL} –- prEN 16154
    \item M/408: \acf{GBAS} --- Cat. I precision approach operations; \ac{APV} – \ac{LPV}
    \item M/510: Aerodrome mapping data
    \item M/524: \acf{ATM} interoperability for the ATM Master Plan
\end{itemize}

Der Inhalt der Mandate sowie die fortlaufende Arbeit der ESOs wird laufend überprüft und vollentwickelte Standards werden mit ihrer Fertigstellung und Abnahme veröffentlicht.
