    Die oben dargelegten \ac{SES}-Rahmenbedingungen werden durch eine Ansammlung von spezifischeren Durchführungsbestimmungen\footnote{engl. \textit{implementing regulations} (IR)} komplementiert.
    Diese sollen die abstrakten Ziele und Grundsätze des Frameworks in detaillierte, anwendungsspezifische Anforderungen übertragen und die Einführung neuer und validierter Betriebskonzepte und Technologien koordinieren \cite[22]{baf_iop}.
    Die einzelnen  Verordnungen werden dabei von der Europäischen Kommission, in Zuarbeit von dem Single Sky Ausschuss, entwickelt und in Komitologie als \acf{DVO} erlassen.
    Die Rahmenverordnung (\vo{VO}{EG}{549/2004}) bestimmt die Funktion des Ausschusses in Art. 5 sowohl nach Beratungsverfahren\footnote{nach Beschluss 1999/468/EG Art. 3 und 7 unter Beachtung des Art. 8 \cite{31999D0468}} als auch nach Regelungsverfahren\footnote{nach Beschluss 1999/468/EG Art. 5 und 7 unter Beachtung des Art. 8 \cite{31999D0468}}.
    Die Ausarbeitungen der Durchführungsvorschriften auf Basis der definierten Rahmenverordnungen \cite[Art. 3]{2004R0549} fällt in die Zuständigkeit von Eurocontrol. 
    Hiernach ist Eurocontrol angehalten, auf Basis von Aufträgen\footnote{in Form von Mandaten} der Kommission -- und in Zusammenhang mit sinnvollen Anhörungen Beteiligter -- die darin beschriebenen Arbeiten innerhalb des angegebenen Zeitrahmens umzusetzen. 
    Im Anschluss wird die Kommission im Rahmen des oben referenzierten Beratungsverfahrens tätig.
    \cite[Art. 8. Abs. 1]{2004R0549}

    \medskip
    Auf der Grundlage dieser erarbeiteten Ergebnisse werden mittels des ebenfalls referenzierten Regelungsverfahrens Entscheidungen über deren Anwendung in der Gemeinschaft und die Fristen von deren Umsetzung getroffen und anschließend im \ac{OJ} veröffentlicht.
    \cite[Art. 8 Abs. 2]{2004R0549}
    
    \bigskip\noindent
    Der Anhang \ref{extra_implementing_regulations} beinhaltet eine Liste aller relevanten Durchführungsverordnungen, welche in Bezug auf die einzelnen Rahmenverordnungen des \ac{SES}-Pakets I definiert wurden.
