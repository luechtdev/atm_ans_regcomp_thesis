\chapter{Rechtliche Rahmenbedingungen}

    \section{Begriffsdefinitionen}
        \graffito{Für den Part 2.1 würde ich mir noch ein wenig Feedback / Verbesserungsvorschläge wünschen}

\subsection{ATM/ANS}\label{beg:atmans}

Air Traffic Managament/Air Navigation Services (\atmans), definiert in \vo{VO}{EU}{2018/1139} Art. 3 Abs. 5, bezeichnet \acf{ATM} und \acf{ANS} und deckt dessen Funktionen und Dienste im Sinne der Rahmenverordnung (siehe \ref{er_549}) ab.
% Darüber hinaus 
Durch den rechtlichen Rahmen der \vo{DVO}{EU}{2017/373} und der verantwortlichen Behörde wurde die DFS für die Erbringung folgender \atmans{} Dienste im deutschen Zuständigkeitsbereich zertifiziert: \cite[17]{ba_technik}
\begin{itemize}
    \item \acf{ATS}, bestehend aus
    \begin{itemize}
        \item \acf{ATC},
        \item Flugalarmdienst,
        \item Fluginformationsdienst;
    \end{itemize}
    \item  Kommunikationsdienst (C);
    \item  Navigationsdienst (N);
    \item  Überwachungsdienst (S);
    \item  \acf{AIS};
    \item  \acf{ATFM};
    \item  \acf{ASM};
    \item  \acf{FPD}
\end{itemize}

\subsection{ATM/ANS Equipment}

ATM/ANS-Equipment (oder auch ATM/ANS-Ausrüstung), definiert in der \acf{DVO} \acs{EU} 2023/1769 Art. 2 Abs. 1, bestimmt Systeme und Komponenten\footnote{jeweils definiert in VO EU 2018/1139 Art. 3 Abs. 6f}, die zur Erbringung der Dienste im funktionalen System beitragen.

\subsection{Funktionales System}

Ein \textit{Funktionales System} (engl. functional system) beschreibt nach der \vo{DVO}{EU}{2017/373} ,,eine Kombination von Verfahren, Personal und Ausrüstung, einschließlich Hardware und Software, zur Erfüllung einer Funktion im Bereich \acs{ATM}/\acs{ANS} und sonstiger Funktionen des Flugverkehrsmanagementnetzes``.
\cite[Anh. I Abs. 56]{2017R0373}
Die \acf{DFS} wird im Rahmen der in \ref{beg:atmans} definierten Dienste auch als Gesamt-Funktionales System (auch \textit{Gesamtsystem Flugsicherung}) qualifiziert.
\cite[17]{ba_technik}

% Im Kontext \vo{DVO}{EU}{2017/373} wird die \acf{DFS} zugleich als Anknüpfungspunkt für die Definition eines Gesamt-Funktionalen Systems einer Organisation (auch \textit{Gesamtsystem Flugsicherung}), wobei jeder zertifizierte Dienst als eigenes sogenanntes \textit{funktionales System} dargestellt wird.

\graffito{Bedürfen noch mehr Begriffe in Bezug auf diesen Abschnitt eine Erläuterung? uU der Begriff "{}Anforderung"}
        \pagebreak
        
    \section{Europäische Rahmenbedingungen}
        \subsection{Single European Sky (\acsfont{SES})}
            \acused{SES}
            
\begin{quote}
,,Der Luftraum über Europa ist – historisch bedingt – stark fragmentiert``:  Er ist in 679 Flugsicherungssektoren aufgeteilt, die von 38 Flugsicherungsorganisationen (ATM) mit jeweils unterschiedlichen
Betriebssystemen10 und Verfahrensabläufen in 65 Bezirkskontrollstellen überwacht werden. Diese Fragmentierung verursacht Kosten, die vermieden werden können:
\cite[S.6]{eu_ses_studie} 
\end{quote}

\noindent
Auch wenn der europäische Luftraum, im Vergleich zu dem flächenmäßig ähnlichen US-amerikanischen Luftraum, (60 \%)\footnote{Stand 2010 (EUROCONTROL)} weniger insgesamt Flüge kontrolliert, so erzeugt die Organisation und Struktur des europäischen Luftraumes (Probleme und Umstände??)
Flugzeuge passieren so im Schnitt deutlich mehr unterschiedliche Center, unterschiedlicher ANSP. 
\cite[S. 74]{eu_ses_fab}

Basierend auf diesen Gründen und Problemen, hat der Europäische Rat am 23. und 24. März 2000 die Kommission auf seiner Sondertagung aufgerufen, zusammen mit Stakeholdern militärischer sowie ziviler Luftfahrt Vorschläge für eine Entsprechende regulative Umsetzung zu erarbeiten.
Die Ergebnisse dieser Arbeit wurden im November des gleichen Jahres vorgelegt.
\cite[ErwG. 2]{2004R0549}

\subsubsection{Zielsetzung}

Die \acf{SES} Initiative der \acf{EU}, dessen Rahmenverordnung 2004 in Kraft trat, zielt ab, Sicherheitsstandards, Ka\-pazitäts- und Umweltaspekte sowie die Gesamteffizienz in dem europäischen Luftverkehrsmanagement zu verbessern.
Der europäische Luftraum soll sich so zu einem ausgewogenem, integrierten Luftraum entwickeln, in dem nationale und territoriale Grenzen keinen negativen Einfluss auf die Abwicklung des Luftverkehrs nehmen und den Anforderungen aller Luftraumnutzern entsprochen wird. \cite[Art. 1 Abs. 1]{2004R0549}
Flugsicherungsorganisationen (\acsp{ANSP}) sollen überregional und übernational zusammenarbeiten und neue Technologien sollen im Rahmen des \acf{SESAR} sollen digitalere und stärker vernetzte \acs{ATM} Systeme der teilnehmenden \acp{ANSP} ermöglichen.
Weiter ist \acs{SES} ein Schlüsselfaktor, um die Umweltauswirkungen der Luftfahrt zu durch effizientere Routen zu reduzieren.  

Das \acs{EU} Parlament schätzt\footnote{aus offizieller, unbekannter Quelle}, dass \ac{SES} zu seiner vollständigen Umsetzung (2030--2035) die europäische ,,Luftraumkapazität verdreifachen und die Kosten des \ac{ATM} halbieren, die Sicherheit um ein Zehnfaches verbessern und die [Umweltauswirkungen] der Luftfahrt [...] um 10 \% verringern [kann].`` 
\cite{eu_parl_ses} 


\subsubsection{Rahmenverordnung}

Im Anschluss dieser Arbeit wurde das sogenannte \acs{SES}-Paket II durch das Europäische Parlament und den Rat beschlossen\footnote{Beschluss nach Art. 100 Abs. 2 AEUV (ehem. Art. 80 Abs. 2 EBV)}.
Das Gesetzespaket umfasst 

\begin{itemize}
    \item einem Vorschlag zur Rahmenverordnung ,,Maßnahmen zur Schaffung eines einheitlichen europäischen Luftraums``\cite{kom_01_564, kom_01_123},
    %\footnote{Mitteilung KOM(2001) 123 und KOM(2001) 564},
    \item einem Vorschlag zu einer Verordnung ,,für eine sichere und effizientere Erbringung von Flugsicherungsdiensten``\cite{kom_01_564_1}; in Einheit mit
    \item  einem Vorschlag zu einer Verordnung ,,zur Ordnung und Nutzung des Luftraums im einheitlichen europäischen Luftraum``\cite{kom_01_564_2}; in Einheit mit
    \item  einem Vorschlag zu einer Verordnung ,,zur Verwirklichung der Interoperabilität des europäischen Flugverkehrsmanagements``\cite{kom_01_564_3}
\end{itemize}

Die Gesetzesvorschläge zu den vier genannten Verordnungen wurden am 10.März 2004 als Verordnungen 549ff./2004 beschlossen, traten am 21.Oktober 2004 in Kraft \cite[S.12]{eu_ses_studie} und definieren die sogenannten Rahmenverordnungen (engl. \acf{ER}) des \ac{SES} Rahmens.

Im Weiteren soll diese Rahmenverordnung als oberste Abstraktionsebene im \ac{SES} Modell dienen. 
Die hierbei definierten Anforderungen und Rahmenbedingungen werden folglich durch spezifischere \acf{IR} und technische \acf{CS} weiter ausgearbeitet, um eine einheitliche Umsetzung der Rahmenbedingungen zu gewährleisten.


\paragraph{Verordnung (EG) Nr. 549/2004} \label{er_549}
    \begin{quote}
        \hspace{-2mm},,Rahmen für die Schaffung eines einheitlichen europäischen Luftraums``\cite{2004R0549, 2004R0549_summary}
    \end{quote}
Die Verordnung 549 wird auch als Rahmenverordnung des SES-Pakets-1 bezeichnet.
Wesentliche Eckpunkte der Verordnung beinhalten:
    \begin{itemize}
        \item die Einrichtung und Verwaltung nationaler Aufsichtsbehörden,
        \item die Einberufung eines Ausschusses für den einheitlichen Luftraum zur Unterstützung der Kommission,
        \item die Erstellung von Durchführungsbestimmungen durch EUROCONTROL auf Basis von Aufträgen des Ausschusses und
        \item    das Ermöglichen von Leistungsüberprüfung- und Sicherheitsmechanismen der einzelnen Staaten. \cite{2004R0549_summary}
    \end{itemize}

\pagebreak
        
\paragraph{Verordnung (EG) Nr. 550/2004} \label{er_550}
\begin{quote}
    ,,Einheitlicher europäischer Luftraum – EU-Bestimmungen für Flugsicherungsdienste``\cite{2004R0550, 2004R0550_summary}
\end{quote}
    Die Verordnung 550 legt gemeinsame Anforderungen für die Erbringung den Flugsicherungsdiensten für den allgemeinen Flugverkehr in der Europäischen Union fest. 
Wesentliche Eckpunkte der Verordnung beschreiben:
    \begin{itemize}
        \item Anforderungen an die Zertifizierung von Flugsicherungsorganisationen,
        \item Aufgaben nationaler Aufsichtsbehörden und deren Autorität über den lokalen Dienstleister; sowie
        \item die mögliche Einrichtung und Voraussetzungen funktionaler Luftraumblöcke(FAB). \cite{2004R0550_summary}
    \end{itemize}

\paragraph{Verordnung (EG) Nr. 551/2004} \label{er_551}
\begin{quote}
    Flugverkehrsmanagement: Ordnung und Nutzung des Luftraums im einheitlichen europäischen Luftraum  \cite{2004R0551, 2004R0551_summary}
\end{quote}
Das Ziel der Verordnung 551 in dem Kontext des SES-Pakets ist es, eine optimale Nutzung des gemeinsamen Luftraums zu gewährleisten und so die Auswirkungen von Flugverspätungen, angesichts des zunehmenden Luftverkehrs, zu minimieren. 
Wichtige Eckpunkte der Verordnung beinhalten: 
    \begin{itemize}
        \item die Schaffung eines einheitlichen europäischen Luftraums inkl. der übernationalen Regelung von Schwankungen der Flugverkehrskapazität;
        \item  einer besseren Integration zwischen militärischem und zivilem Luftraum für eine optimierte flexible Luftraumnutzung,
        \item ein besseres Netzmanagement für die optimale Nutzung funktionaler Luftraumblöcke im Interesse der Luftraumnutzer. \cite{2004R0551_summary}
    \end{itemize}

\pagebreak

\paragraph{Verordnung (EG) Nr. 552/2004} \label{er_552}

\begin{quote}
    Interoperabilität des europäischen Flugverkehrsmanagement\-netzes\cite{2004R0552, 2004R0552_summary}
\end{quote}
Die Verordnung 552 -- auch genannt ,,Interoperabilitätsverordnung`` -- zielt ab, gemeinsame Anforderungen für die verschiedenen nationalen Flugverkehrsmanagementsysteme festzulegen, um eine Interoperabilität aller Systeme sicherzustellen. \cite{2004R0552_summary}
Hierbei definiert die Verordnung ein einheitliches System für die Zertifizierung von Komponenten und Systemen\footnote{darunter ATM/ANS-Komponenten / -Systeme} sowie den dazugehörigen Verfahren und stellt zudem sicher, dass neue zugelassene und validierte Betriebskonzepte eingeführt werden. \cite[Art. 3 Abs. 1]{2004R0552}

Die Flugsicherungen(ANSP) erklären folglich EG-Prüferklärungen\footnote{Gültigkeit bis 2023: Ablösung durch SoC unter EASA nach EU VO 1769/2023 \cite{2023R1769}}, welche der nationalen Aufsichtsbehörde\footnote{Bundesamt für Flugsicherung(BAF) in Deutschland} vorgelegt werden. \cite[Art. 6 Abs. 1f]{2004R0552}


    \begin{itemize}
        \item Grundlegende Anforderungen
        \item Maßnahmen im Bereich der Interoperabilität
        \item Gemeinschaftliche Spezifikationen
        \item Komponenten EGK / EGG
        \item Schutzmaßnahmen
        \item Übergangsbestimmungen
    \end{itemize}

\paragraph{Verordnung (EU) Nr. 2018/1139}

\begin{quote}
    ,,Festlegung gemeinsamer Vorschriften für die Zivilluftfahrt und zur Errichtung einer Agentur der Europäischen Union für Flugsicherheit [...] und zur Aufhebung der Verordnungen (EG) Nr. 552/2004 [...]`` \cite{2018R1139} 
\end{quote}

Die Verordnung 1139\footnote{Nummerierung harmonisiert nach 2015\cite{eu_number_harmony}} aus dem Jahre 2018 


    \pagebreak


% \subsubsection{Kontrolle und Einhaltung}

% Die Einhaltung der entsprechenden verabschiedeten Rahmenbedingungen und Anforderungen unterliegen aufgrund des Trennungsgebots § 31b Abs. 1 Satz 1 des \ac{LuftVG} i.V.m. Art. 87 d Abs. 1 \ac{GG} einer ,,funktional unabhängigen``\footnote{Art. 4 Abs. 1f} Behörde. \cite[S. 14]{eu_ses_studie} 
% In Deutschland übernimmt diese Aufgabe das \acf{BAF}.

            % \pagebreak
            
        \subsection{Durchführungsvorschriften (\acsfont{IR})}
            \acused{IR}
            \begin{quote}
\textcolor{red}{Erklärung der Funktion von Implementing Regulations nach SES Definition u.U. Beispiele}
\end{quote}
\begin{quote}
\textcolor{green}{1 Seite}
\end{quote}


                \pagebreak
        
        \subsection{Gemeinschaftliche Spezifikationen (\acsfont{CS})}
            \acused{CS}
            \begin{quote}
\textcolor{red}{Erklärung der Funktion von Community Specifications (EUROCONTROL etc)}
\end{quote}
\begin{quote}
\textcolor{green}{1 Seite sollte reichen}
\end{quote}


                \pagebreak

    \section{Sonstige internationale Rahmenbedingungen}

Neben den oben beschriebenen europäischen Regularien und Anforderungen an \atmans-Equipment wurde bereits seid den Anfängen der Zivilluftfahrt verstanden, dass standardisierte Abläufe und Spezifikationen ein unabdingbares Mittel für eine weltweit einheitliche Luftverkehrssicherheit darstellen.
\cite[156]{eu_icao_milde}
Auf dieser Basis bestehen bereits viele internationale Rahmenbedingungen, welche unter anderem auch wesentliche Spezifikationen und Anforderungen von \atmans-Equipment definieren.
Sämtliche Anforderungen, bspw. für den europäischen Raum in \ac{SES} definiert, beziehen sich so auf die gleiche Basis international einheitlich definierter Standards.

\subsubsection{ICAO}
Die \acf{ICAO} ist eine Sonderorganisation der \acf{UN}, welche mit dem Abkommen über die Internationale Zivilluftfahrt (auch Chicagoer Abkommen) gegründet wurde und fortan Standards, Empfehlungen und Richtlinien für die zivile Luftfahrt erlässt. 
Diese bieten eine wichtige Grundlage als Referenz von definierten Anforderungen. 
Beispielsweise in den ,,Arbeitsmethodiken und Betriebsverfahren für die Erbringung von Flugberatungsdiensten`` (\textsf{AIS.TR.100}) aus der \vo{DVO}{EU}{2017/373} welche sich auf die \ac{ICAO} Anhänge 4 (\textit{Aeronautical Charts}) und 15 (\textit{Aeronautical Information Services}) beziehen.
\cite[Anh. IV]{2017R0373}

\subsubsection{SARP / PANS}


% \graffito{Hier schreib ich bis morgen noch fertig}
% Standards And Recommended Practices (SARPs) are technical specifications adopted by the Council of ICAO in accordance with Article 38 of the Convention on International Civil Aviation in order to achieve "the highest practicable degree of uniformity in regulations, standards, procedures and organization in relation to aircraft, personnel, airways and auxiliary services in all matters in which such uniformity will facilitate and improve air navigation".
% SARPs are published by ICAO in the form of Annexes to Chicago Convention. SARPs do not have the same legal binding force as the Convention itself, because Annexes are not international treaties. Moreover States agreed to "undertake to collaborate in securing (...) uniformity", not to "comply with".[1] Each Contracting State may notify the ICAO Council of differences between SARPs and its own regulations and practices.[2] Those differences are published in the form of Supplements to Annexes. 

Internationale \acf{SARP} beschreiben die spezifischen technische Spezifikationen, welche in den \ac{ICAO} Anhängen definiert sind.
Sie umfassen alle in Art. 37 a--k beschriebenen Themenbereiche, darunter auch \atmans.  \cite[S. 43f.]{icao_convention}
% nach Art. 38 der \ac{ICAO} Convention von dem \ac{ICAO}-Council angenommen wurden.
% \acp{SARP} werden in Form von Anhängen
Da die Anhänge selber rechtlich keinen Stellenwert internationaler Abkommen haben, fehlt diesen die unmittelbare Rechstverbindlichkeit. 
\cite[156--159]{eu_icao_milde}
Eine abweichende Nichteinhaltung gilt es dem Council nach Art. 38 mitzuteilen. \cite[S. 45f.]{icao_convention}

\ac{SARP} werden durch \ac{PANS} ergänzt, diese beschreiben genaue Verfahren, welche für Anwendungen der Flugsicherung und dessen Flugverkehrsdienste implementiert werden können. 


\subsubsection{EUROCAE}
Die \acf{EUROCAE}\footnote{\href{https://www.eurocae.net/}{https://www.eurocae.net/}} ist eine gemeinnützige Organisation der Europäischen Union, welche \acf{CS} für \atmans-Produkte entwickelt. 
\ac{EUROCAE} arbeitet hierbei eng mit der oben genannten\ac{ICAO}, anderen beteiligten Organisationen, Herstellern, Betreiber und weiteren Luftfahrt-Stakeholder:innen zusammen, um das Erschaffen von harmonisierten Standards zu fördern.

