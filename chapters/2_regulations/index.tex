\chapter{Rechtliche Rahmenbedingungen}

    \section{Begriffsdefinitionen}
        \subsection{ATM/ANS}

Air Traffic Managament/Air Navigation Services (ATM/ANS), definiert in Verordnung EU 2018/1139 Art. 3 Abs. 5, bezeichnet Flugverkehrsmanagement (ATM) und Flugsicherungsdienste (ANS) und deckt dessen Funktionen und Dienste im Sinne der Rahmenverordnung (siehe \ref{er_549}) ab.
Darüber hinaus 


Im Kontext der Deutschen Flugsicherung (DFS) dient der Begriff zugleich als Anknüpfungspunkt für die Definition eines Gesamt-Funktionalen Systems einer Organisation (auch \textit{Gesamtsystem Flugsicherung}), wobei jeder zertifizierte Dienst als eigenes sogenanntes \textit{funktionales System} dargestellt wird.
Durch den rechtlichen Rahmen und die verantwortliche Behörde wurde die DFS für die Erbringung folgender Dienste im deutschen Zuständigkeitsbereich zertifiziert: \cite[17]{ba_technik}
\begin{itemize}
    \item  Flugverkehrsdienste (ATS), bestehend aus
    \begin{itemize}
        \item Flugverkehrskontrolldiensten (ATC),
        \item Flugalarmdienst,
        \item Fluginformationsdienst;
    \end{itemize}
    \item  Kommunikationsdienst (C);
    \item  Navigationsdienst (N);
    \item  Überwachungsdienst (S);
    \item  Flugberatungsdienst (AIS);
    \item  Verkehrsflussregelung (ATFM);
    \item  Luftraummanagement (ASM);
    \item  Flugverfahrensplanung (FPD)
\end{itemize}

\subsection{ATM/ANS Equipment}

ATM/ANS-Equipment (oder auch ATM/ANS-Ausrüstung), definiert in der Durchführungsverordnung EU 2023/1769 Art. 2 Abs. 1, bestimmt Systeme und Komponenten\footnote{jeweils definiert in VO EU 2018/1139 Art. 3 Abs. 6f}, die zur Erbringung der oben genannten Definition beitragen.
        \pagebreak

        
    \section{Europäische Rahmenbedingungen}
        \subsection{Single European Sky (\acsfont{SES})}
            
    \begin{quote}
        ,,Der Luftraum über Europa ist – historisch bedingt – stark fragmentiert``:  Er ist in 679 Flugsicherungssektoren aufgeteilt, die von 38 Flugsicherungsorganisationen (ATM) mit jeweils unterschiedlichen
        Betriebssystemen und Verfahrensabläufen in 65 Bezirkskontrollstellen überwacht werden. Diese Fragmentierung verursacht Kosten, die vermieden werden können:
        \cite[S.6]{eu_ses_studie} 
    \end{quote}
        
    \noindent
    Auch wenn der europäische Luftraum, im Vergleich zu dem flächenmäßig ähnlichen US-amerikanischen Luftraum insgesamt (60 \%)\footnote{Stand 2010 (Eurocontrol)} weniger Flüge kontrolliert, so erzeugen die vielen beteiligten Flugischerungs-Organisatio\-nen aufgrund der staatlich orientierten Strukturen des Europäischen Luftraumes\footnote{vgl. Motivation des Abkommens über die Internationale Zivilluftfahrt \cite{icao_convention}} erhebliche Einschränkungen in puncto Effizienz, Kosten und Performance.
    Flugzeuge passieren so im Schnitt deutlich mehr Luftraumblöcke verschiedener \ac{ANSP}, was zu komplizierten Prozessen der Abrechnung und ineffizienten Flugrouten führt.  
    \cite[S. 74]{eu_ses_fab}

    \medskip
    Basierend auf diesen Gründen und Problemen, hat der Europäische Rat am 23. und 24. März 2000 die Kommission auf seiner Sondertagung aufgerufen, zusammen mit Stakeholdern militärischer sowie ziviler Luftfahrt Vorschläge für eine entsprechende regulative Umsetzung zu erarbeiten.
    Die Ergebnisse dieser Arbeit wurden im November des gleichen Jahres vorgelegt.
    \cite[ErwG. 2]{2004R0549}
    
\subsubsection{Zielsetzung}

    Die \acf{SES} Initiative der \acf{EU}, deren Rahmenverordnung 2004 in Kraft trat, zielt darauf ab, Sicherheitsstandards, Ka\-pazitäts- und Umweltaspekte sowie die Gesamteffizienz im europäischen Luftverkehrsmanagement zu verbessern.
    Der europäische Luftraum soll sich so zu einem ausgewogenem, integrierten Luftraum entwickeln, in dem nationale und territoriale Grenzen keinen negativen Einfluss auf die Abwicklung des Luftverkehrs nehmen und den Anforderungen aller Luftraumnutzern entsprochen wird 
    \cite[Art. 1 Abs. 1]{2004R0549}.
    Flugsicherungsorganisationen (\acs{ANSP}) sollen überregional und übernational zusammenarbeiten und neue Technologien sollen im Rahmen des \acf{SESAR} digitalere und stärker vernetzte \acs{ATM} Systeme der teilnehmenden \ac{ANSP} ermöglichen.
    Weiter ist \acs{SES} ein Schlüsselfaktor, um die Umweltauswirkungen der Luftfahrt durch effizientere Routen zu reduzieren.  

    \medskip
    Das \acs{EU} Parlament schätzt\footnote{aus den ,,,offiziellen` Ziele(n) des \ac{SES}, deren Ursprung unklar ist``\cite{eu_parl_ses}}, dass \ac{SES} zu seiner vollständigen Umsetzung (2030--2035) die europäische ,,Luftraumkapazität verdreifachen und die Kosten des \ac{ATM} halbieren, die Sicherheit um ein Zehnfaches verbessern und die [Umweltauswirkungen] der Luftfahrt [...] um 10 \% verringern [kann].`` 
    \cite{eu_parl_ses} 


\subsubsection{Rahmenverordnung}

    Im Anschluss dieser Zielsetzung wurde das sogenannte \acs{SES}-Paket I durch das Europäische Parlament und den Rat beschlossen\footnote{Beschluss nach Art. 100 Abs. 2 AEUV (ehem. Art. 80 Abs. 2 EBV)}.
    Das Gesetzespaket umfasst 
    
    \begin{itemize}
        \item einen Vorschlag zur Rahmenverordnung ,,Maßnahmen zur Schaffung eines einheitlichen europäischen Luftraums``\cite{kom_01_564, kom_01_123},
        %\footnote{Mitteilung KOM(2001) 123 und KOM(2001) 564},
        \item einen Vorschlag zu einer Verordnung ,,für eine sichere und effizientere Erbringung von Flugsicherungsdiensten``\cite{kom_01_564_1}; in Einheit mit
        \item  einen Vorschlag zu einer Verordnung ,,zur Ordnung und Nutzung des Luftraums im einheitlichen europäischen Luftraum``\cite{kom_01_564_2}; in Einheit mit
        \item  einen Vorschlag zu einer Verordnung ,,zur Verwirklichung der Interoperabilität des europäischen Flugverkehrsmanagements``\cite{kom_01_564_3}.
    \end{itemize}

    \medskip
    Die Gesetzesvorschläge zu den vier genannten Verordnungen wurden am 10.März 2004 als Verordnungen \acs{VO} (\acs{EG}) \acsfont{Nrn.} 549ff./2004 beschlossen und traten am 21.Oktober 2004 in Kraft \cite[S.12]{eu_ses_studie}.
    Sie definieren die sogenannten Rahmenverordnungen (engl. \acf{ER}) des \ac{SES} Rahmens.

    \medskip
    Im Weiteren sollen diese Rahmenverordnungen als oberste Abstraktionsebene im \ac{SES}-Modell dienen. 
    Die hierbei definierten Anforderungen und Rahmenbedingungen werden folglich durch spezifischere \acf{IR} und technische \acf{CS} weiter ausgearbeitet, um eine einheitliche Umsetzung der Rahmenbedingungen zu gewährleisten.
    Die Rahmenverordnungen gewährleisten hierbei die Integrität der definierten Ziele im Rahmen aller im Folgenden formulierten Dokumente.
    
    
\paragraph{Verordnung (EG) Nr. 549/2004} \label{er_549}
    \begin{quote}
        \hspace{-2mm},,Rahmen für die Schaffung eines einheitlichen europäischen Luftraums``\cite{2004R0549, 2004R0549_summary}
    \end{quote}
Die Verordnung 549 wird auch als Rahmenverordnung des SES-Pakets-1 bezeichnet.
Wesentliche Eckpunkte der Verordnung beinhalten:
    \begin{itemize}
        \item die Einrichtung und Verwaltung nationaler Aufsichtsbehörden,
        \item die Einberufung eines Ausschusses für den einheitlichen Luftraum zur Unterstützung der Kommission,
        \item die Erstellung von Durchführungsbestimmungen durch EUROCONTROL auf Basis von Aufträgen des Ausschusses und
        \item    das Ermöglichen von Leistungsüberprüfung- und Sicherheitsmechanismen der einzelnen Staaten. \cite{2004R0549_summary}
    \end{itemize}

\pagebreak
        
\paragraph{Verordnung (EG) Nr. 550/2004} \label{er_550}
\begin{quote}
    ,,Einheitlicher europäischer Luftraum – EU-Bestimmungen für Flugsicherungsdienste``\cite{2004R0550, 2004R0550_summary}
\end{quote}
    Die Verordnung 550 legt gemeinsame Anforderungen für die Erbringung den Flugsicherungsdiensten für den allgemeinen Flugverkehr in der Europäischen Union fest. 
Wesentliche Eckpunkte der Verordnung beschreiben:
    \begin{itemize}
        \item Anforderungen an die Zertifizierung von Flugsicherungsorganisationen,
        \item Aufgaben nationaler Aufsichtsbehörden und deren Autorität über den lokalen Dienstleister; sowie
        \item die mögliche Einrichtung und Voraussetzungen funktionaler Luftraumblöcke(FAB). \cite{2004R0550_summary}
    \end{itemize}

\paragraph{Verordnung (EG) Nr. 551/2004} \label{er_551}
\begin{quote}
    Flugverkehrsmanagement: Ordnung und Nutzung des Luftraums im einheitlichen europäischen Luftraum  \cite{2004R0551, 2004R0551_summary}
\end{quote}
Das Ziel der Verordnung 551 in dem Kontext des SES-Pakets ist es, eine optimale Nutzung des gemeinsamen Luftraums zu gewährleisten und so die Auswirkungen von Flugverspätungen, angesichts des zunehmenden Luftverkehrs, zu minimieren. 
Wichtige Eckpunkte der Verordnung beinhalten: 
    \begin{itemize}
        \item die Schaffung eines einheitlichen europäischen Luftraums inkl. der übernationalen Regelung von Schwankungen der Flugverkehrskapazität;
        \item  einer besseren Integration zwischen militärischem und zivilem Luftraum für eine optimierte flexible Luftraumnutzung,
        \item ein besseres Netzmanagement für die optimale Nutzung funktionaler Luftraumblöcke im Interesse der Luftraumnutzer. \cite{2004R0551_summary}
    \end{itemize}

\pagebreak

\paragraph{Verordnung (EG) Nr. 552/2004} \label{er_552}

\begin{quote}
    Interoperabilität des europäischen Flugverkehrsmanagement\-netzes\cite{2004R0552, 2004R0552_summary}
\end{quote}
Die Verordnung 552 -- auch genannt ,,Interoperabilitätsverordnung`` -- zielt ab, gemeinsame Anforderungen für die verschiedenen nationalen Flugverkehrsmanagementsysteme festzulegen, um eine Interoperabilität aller Systeme sicherzustellen. \cite{2004R0552_summary}
Hierbei definiert die Verordnung ein einheitliches System für die Zertifizierung von Komponenten und Systemen\footnote{darunter ATM/ANS-Komponenten / -Systeme} sowie den dazugehörigen Verfahren und stellt zudem sicher, dass neue zugelassene und validierte Betriebskonzepte eingeführt werden. \cite[Art. 3 Abs. 1]{2004R0552}

Die Flugsicherungen(ANSP) erklären folglich EG-Prüferklärungen\footnote{Gültigkeit bis 2023: Ablösung durch SoC unter EASA nach EU VO 1769/2023 \cite{2023R1769}}, welche der nationalen Aufsichtsbehörde\footnote{Bundesamt für Flugsicherung(BAF) in Deutschland} vorgelegt werden. \cite[Art. 6 Abs. 1f]{2004R0552}


    \begin{itemize}
        \item Grundlegende Anforderungen
        \item Maßnahmen im Bereich der Interoperabilität
        \item Gemeinschaftliche Spezifikationen
        \item Komponenten EGK / EGG
        \item Schutzmaßnahmen
        \item Übergangsbestimmungen
    \end{itemize}

\paragraph{Verordnung (EU) Nr. 2018/1139}

\begin{quote}
    ,,Festlegung gemeinsamer Vorschriften für die Zivilluftfahrt und zur Errichtung einer Agentur der Europäischen Union für Flugsicherheit [...] und zur Aufhebung der Verordnungen (EG) Nr. 552/2004 [...]`` \cite{2018R1139} 
\end{quote}

Die Verordnung 1139\footnote{Nummerierung harmonisiert nach 2015\cite{eu_number_harmony}} aus dem Jahre 2018 


    \pagebreak


            \pagebreak
            
        \subsection{Implementing Rules (IR)}
                Die oben dargelegten \ac{SES}-Rahmenbedingungen werden durch eine Ansammlung von spezifischeren Durchführungsbestimmungen\footnote{engl. \textit{implementing regulations} (IR)} komplementiert.
    Diese sollen die abstrakten Ziele und Grundsätze des Frameworks in detaillierte, anwendungsspezifische Anforderungen übertragen und die Einführung neuer und validierter Betriebskonzepte und Technologien koordinieren \cite[22]{baf_iop}.
    Die einzelnen  Verordnungen werden dabei von der Europäischen Kommission, in Zuarbeit von dem Single Sky Ausschuss, entwickelt und in Komitologie als \acf{DVO} erlassen.
    Die Rahmenverordnung (\vo{VO}{EG}{549/2004}) bestimmt die Funktion des Ausschusses in Art. 5 sowohl nach Beratungsverfahren\footnote{nach Beschluss 1999/468/EG Art. 3 und 7 unter Beachtung des Art. 8 \cite{31999D0468}} als auch nach Regelungsverfahren\footnote{nach Beschluss 1999/468/EG Art. 5 und 7 unter Beachtung des Art. 8 \cite{31999D0468}}.
    Die Ausarbeitungen der Durchführungsvorschriften auf Basis der definierten Rahmenverordnungen \cite[Art. 3]{2004R0549} fällt in die Zuständigkeit von Eurocontrol. 
    Hiernach ist Eurocontrol angehalten, auf Basis von Aufträgen\footnote{in Form von Mandaten} der Kommission -- und in Zusammenhang mit sinnvollen Anhörungen Beteiligter -- die darin beschriebenen Arbeiten innerhalb des angegebenen Zeitrahmens umzusetzen. 
    Im Anschluss wird die Kommission im Rahmen des oben referenzierten Beratungsverfahrens tätig.
    \cite[Art. 8. Abs. 1]{2004R0549}

    \medskip
    Auf der Grundlage dieser erarbeiteten Ergebnisse werden mittels des ebenfalls referenzierten Regelungsverfahrens Entscheidungen über deren Anwendung in der Gemeinschaft und die Fristen von deren Umsetzung getroffen und anschließend im \ac{OJ} veröffentlicht.
    \cite[Art. 8 Abs. 2]{2004R0549}
    
    \bigskip\noindent
    Der Anhang \ref{extra_implementing_regulations} beinhaltet eine Liste aller relevanten Durchführungsverordnungen, welche in Bezug auf die einzelnen Rahmenverordnungen des \ac{SES}-Pakets I definiert wurden.

            \pagebreak
        
        \subsection{Community Specifications (CS)}
            
    \acf{CS} bilden die spezifischste, dritte Abstraktionsebene des \ac{SES}-Rahmens.
    Sie stehen außerhalb des verpflichtenden Teils der Regularien und dessen Anwendung ist für alle teilnehmenden Parteien freiwillig. 
    Generell beschreiben \ac{CS}
    
    \begin{itemize}
        \item einen Standard für Systeme oder Komponenten, welcher von einer \ac{ESO} wie \ac{CEN}, \ac{CENELEC}, \ac{ETSI} in Kooperation mit \ac{EUROCAE}, auf einem Mandat der Europäischen Kommission herausgegeben wurde, oder
        \item eine Spezifikation bezüglich der operativen Koordination zwischen den \acp{ANSP}, welche auf Ersuchen der Europäischen Kommission von Eurocontrol herausgegeben wurde.
    \end{itemize}

    \noindent
    Angenommene \ac{CS} werden im \ac{OJ} veröffentlicht.
    Beispielsweise:

    \begin{itemize}
        \item \ac{ETSI} EN 303 214 V1.2.1\footnote{\ac{OJ} 2012/C/168/03}: 
            \acf{DLS} System: Requirements for ground constituents and system testing
        \item Eurocontrol Spec-0100 Edition 2.0\footnote{\ac{OJ} 2007/C/188/03}:
            Eurocontrol Specification of Interoperability and performance requirements for the \acf{FMTP}
    \end{itemize}
    
    \noindent
    Neben den bereits veröffentlichten \ac{CS} wird weiterhin von den \acp{ESO} an Standards gearbeitet oder diese initiiert. 
    Diese Arbeit basiert auf den Standardisierungsmandaten der Europäischen Kommission an die \acp{ESO}, welche die Ziele der zu erstellenden \ac{CS} darlegen.
    Diese umfassen die Mandate\footnote{\href{https://portal.etsi.org/EC-EFTA-Mandates}{https://portal.etsi.org/EC-EFTA-Mandates}}:
    \begin{itemize}
        \item M/390: \acf{SWAL} –- prEN 16154
        \item M/408: \acf{GBAS} --- Cat. I precision approach operations; \ac{APV} – \ac{LPV}
        \item M/510: Aerodrome mapping data
        \item M/524: \acf{ATM} interoperability for the ATM Master Plan
    \end{itemize}
    
    Der Inhalt der Mandate sowie die fortlaufende Arbeit der ESOs wird laufend überprüft und vollentwickelte Standards werden mit ihrer Fertigstellung und Abnahme veröffentlicht.
    
                \pagebreak

    \section{Sonstige internationale Rahmenbedingungen}

    \begin{quote}
\textcolor{red}{Wenn sinnvoll: Kurzer Exkurs zu anderen Anforderungen wie ICAO Annex, etc}
\end{quote}
\begin{quote}
\textcolor{green}{1 Seite sollte reichen}
\end{quote}

ATM/ANS Produkte sind Produkte, die für die Flugsicherung und die Flugverkehrsdienste verwendet werden. Sie müssen bestimmten Rahmenbedingungen entsprechen, die von internationalen Organisationen festgelegt werden. Die wichtigsten Rahmenbedingungen sind:

\subsection{ICAO}
- ICAO: Die Internationale Zivilluftfahrt-Organisation ist eine Sonderorganisation der Vereinten Nationen, die Standards und Empfehlungen für die zivile Luftfahrt erlässt. Die ICAO veröffentlicht die Standards and Recommended Practices (SARP), die die Mindestanforderungen für ATM/ANS Produkte definieren.

Weiter bieten Richtlinien der \ac{ICAO} eine wichtige Grundlage für internationale Arbeitsmethodiken und Betriebsverfahren, die aus dem Abkommen von Chicago hervorgehen.
Beispielsweise als Referenz in der Anforderung zu den ,,Arbeitsmethodiken und Betriebsverfahren für die Erbringung von Flugberatungsdiensten``(\textsf{AIS.TR.100}) welche sich auf die ICAO Anhänge 4(\textit{Aeronautical Charts}) und 15(\textit{Aeronautical Information Services}) beziehen.
\cite[Anh. IV]{2017R0373}

\subsection{PANS}
- PANS: Die Procedures for Air Navigation Services sind Dokumente, die von der ICAO herausgegeben werden und die Verfahren für die Flugsicherung und die Flugverkehrsdienste beschreiben. Die PANS ergänzen die SARP und geben praktische Anleitungen für ATM/ANS Produkte.

\subsection{EUROCAE}
- EUROCAE: Die European Organisation for Civil Aviation Equipment ist eine gemeinnützige Organisation, die technische Spezifikationen für ATM/ANS Produkte entwickelt. Die EUROCAE arbeitet eng mit der ICAO und anderen Organisationen zusammen, um harmonisierte Standards zu fördern.

