
\paragraph{Verordnung (EG) Nr. 549/2004} \label{er_549}
    \begin{quote}
        \hspace{-2mm},,Rahmen für die Schaffung eines einheitlichen europäischen Luftraums``\cite{2004R0549, 2004R0549_summary}
    \end{quote}
Die Verordnung 549 wird auch als Rahmenverordnung des SES-Pakets-1 bezeichnet.
Wesentliche Eckpunkte der Verordnung beinhalten:
    \begin{itemize}
        \item die Einrichtung und Verwaltung nationaler Aufsichtsbehörden,
        \item die Einberufung eines Ausschusses für den einheitlichen Luftraum zur Unterstützung der Kommission,
        \item die Erstellung von Durchführungsbestimmungen durch EUROCONTROL auf Basis von Aufträgen des Ausschusses und
        \item    das Ermöglichen von Leistungsüberprüfung- und Sicherheitsmechanismen der einzelnen Staaten. \cite{2004R0549_summary}
    \end{itemize}

\pagebreak
        
\paragraph{Verordnung (EG) Nr. 550/2004} \label{er_550}
\begin{quote}
    ,,Einheitlicher europäischer Luftraum – EU-Bestimmungen für Flugsicherungsdienste``\cite{2004R0550, 2004R0550_summary}
\end{quote}
    Die Verordnung 550 legt gemeinsame Anforderungen für die Erbringung den Flugsicherungsdiensten für den allgemeinen Flugverkehr in der Europäischen Union fest. 
Wesentliche Eckpunkte der Verordnung beschreiben:
    \begin{itemize}
        \item Anforderungen an die Zertifizierung von Flugsicherungsorganisationen,
        \item Aufgaben nationaler Aufsichtsbehörden und deren Autorität über den lokalen Dienstleister; sowie
        \item die mögliche Einrichtung und Voraussetzungen funktionaler Luftraumblöcke(FAB). \cite{2004R0550_summary}
    \end{itemize}

\paragraph{Verordnung (EG) Nr. 551/2004} \label{er_551}
\begin{quote}
    Flugverkehrsmanagement: Ordnung und Nutzung des Luftraums im einheitlichen europäischen Luftraum  \cite{2004R0551, 2004R0551_summary}
\end{quote}
Das Ziel der Verordnung 551 in dem Kontext des SES-Pakets ist es, eine optimale Nutzung des gemeinsamen Luftraums zu gewährleisten und so die Auswirkungen von Flugverspätungen, angesichts des zunehmenden Luftverkehrs, zu minimieren. 
Wichtige Eckpunkte der Verordnung beinhalten: 
    \begin{itemize}
        \item die Schaffung eines einheitlichen europäischen Luftraums inkl. der übernationalen Regelung von Schwankungen der Flugverkehrskapazität;
        \item  einer besseren Integration zwischen militärischem und zivilem Luftraum für eine optimierte flexible Luftraumnutzung,
        \item ein besseres Netzmanagement für die optimale Nutzung funktionaler Luftraumblöcke im Interesse der Luftraumnutzer. \cite{2004R0551_summary}
    \end{itemize}

\pagebreak

\paragraph{Verordnung (EG) Nr. 552/2004} \label{er_552}

\begin{quote}
    Interoperabilität des europäischen Flugverkehrsmanagement\-netzes\cite{2004R0552, 2004R0552_summary}
\end{quote}
Die Verordnung 552 -- auch genannt ,,Interoperabilitätsverordnung`` -- zielt ab, gemeinsame Anforderungen für die verschiedenen nationalen Flugverkehrsmanagementsysteme festzulegen, um eine Interoperabilität aller Systeme sicherzustellen. \cite{2004R0552_summary}
Hierbei definiert die Verordnung ein einheitliches System für die Zertifizierung von Komponenten und Systemen\footnote{darunter ATM/ANS-Komponenten / -Systeme} sowie den dazugehörigen Verfahren und stellt zudem sicher, dass neue zugelassene und validierte Betriebskonzepte eingeführt werden. \cite[Art. 3 Abs. 1]{2004R0552}

Die Flugsicherungen(ANSP) erklären folglich EG-Prüferklärungen\footnote{Gültigkeit bis 2023: Ablösung durch SoC unter EASA nach EU VO 1769/2023 \cite{2023R1769}}, welche der nationalen Aufsichtsbehörde\footnote{Bundesamt für Flugsicherung(BAF) in Deutschland} vorgelegt werden. \cite[Art. 6 Abs. 1f]{2004R0552}


    \begin{itemize}
        \item Grundlegende Anforderungen
        \item Maßnahmen im Bereich der Interoperabilität
        \item Gemeinschaftliche Spezifikationen
        \item Komponenten EGK / EGG
        \item Schutzmaßnahmen
        \item Übergangsbestimmungen
    \end{itemize}

\paragraph{Verordnung (EU) Nr. 2018/1139}

\begin{quote}
    ,,Festlegung gemeinsamer Vorschriften für die Zivilluftfahrt und zur Errichtung einer Agentur der Europäischen Union für Flugsicherheit [...] und zur Aufhebung der Verordnungen (EG) Nr. 552/2004 [...]`` \cite{2018R1139} 
\end{quote}

Die Verordnung 1139\footnote{Nummerierung harmonisiert nach 2015\cite{eu_number_harmony}} aus dem Jahre 2018 


    \pagebreak
