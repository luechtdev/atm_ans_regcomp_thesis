\pagebreak

\paragraph{Verordnung (\acsfont{EG}) Nr. 549/2004} \label{er_549}

    \begin{quote}
        ,,Rahmen für die Schaffung eines einheitlichen europäischen Luftraums``\\
        \cite{2004R0549, 2004R0549_summary}
    \end{quote}

    \noindent
    Die Verordnung Nr. 549 wird auch als Rahmenverordnung des \ac{SES}-Pakets I bezeichnet.
    Wesentliche Eckpunkte der Verordnung beinhalten
    
    \begin{itemize}
        \item die Einrichtung und Verwaltung nationaler Aufsichtsbehörden\footnote{zulande das \acf{BAF}},
        \item die Einberufung eines Ausschusses für den einheitlichen Luftraum zur Unterstützung der Kommission,
        \item die Erstellung von Durchführungsbestimmungen durch Eurocontrol auf Basis von Mandaten des Ausschusses\footnote{siehe \ref{ch:cs}} und
        \item das Ermöglichen von Leistungsüberprüfung- und Sicherheitsmechanismen der einzelnen Staaten. \cite{2004R0549_summary}
    \end{itemize}
        
\paragraph{Verordnung (\acsfont{EG}) Nr. 550/2004} \label{er_550}
    \begin{quote}
        ,,Einheitlicher europäischer Luftraum – EU-Bestimmungen für Flugsicherungsdienste`` 
        \cite{2004R0550, 2004R0550_summary}
    \end{quote}

    \noindent
    Die Verordnung Nr. 550 legt gemeinsame Anforderungen für die Erbringung den Flugsicherungsdiensten für den allgemeinen Flugverkehr in der Europäischen Union fest. 
    Wesentliche Eckpunkte der Verordnung beschreiben:
    
    \begin{itemize}
        \item Anforderungen an die Zertifizierung von Flugsicherungsorganisationen,
        \item Aufgaben nationaler Aufsichtsbehörden und deren Autorität über den lokalen Dienstleister; sowie
        \item die mögliche Einrichtung und Voraussetzungen funktionaler Luftraumblöcke (\acs{FAB}). \cite{2004R0550_summary}
    \end{itemize}

\paragraph{Verordnung (\acsfont{EG}) Nr. 551/2004} \label{er_551}
    \begin{quote}
        ,,Flugverkehrsmanagement: Ordnung und Nutzung des Luftraums im einheitlichen europäischen Luftraum``
        \cite{2004R0551, 2004R0551_summary}
    \end{quote}

    \noindent
    Das Ziel der Verordnung Nr. 551 in dem Kontext des \ac{SES}-Pakets ist es, eine optimale Nutzung des gemeinsamen Luftraums zu gewährleisten und so die Auswirkungen von Flugverspätungen, angesichts des zunehmenden Luftverkehrs, zu minimieren. 
    Wichtige Eckpunkte der Verordnung beinhalten: 
    
    \begin{itemize}
        \item die Schaffung eines einheitlichen europäischen Luftraums inkl. der übernationalen Regelung von Schwankungen der Flugverkehrskapazität;
        \item  einer besseren Integration zwischen militärischem und zivilem Luftraum für eine optimierte flexible Luftraumnutzung,
        \item ein besseres Netzmanagement für die optimale Nutzung funktionaler Luftraumblöcke im Interesse der Luftraumnutzer. \cite{2004R0551_summary}
    \end{itemize}

\paragraph{Verordnung (\acsfont{EG}) Nr. 552/2004} \label{er_552}
    \begin{quote}
        ,,Interoperabilität des europäischen Flugverkehrsmanagement\-netzes``
        \cite{2004R0552, 2004R0552_summary}
    \end{quote}
    
    \noindent
    Die Verordnung Nr. 552 -- auch genannt ,,Interoperabilitätsverordnung`` -- zielt darauf ab, gemeinsame Anforderungen für die verschiedenen nationalen Flugverkehrsmanagementsysteme festzulegen, um eine Interoperabilität aller Systeme sicherzustellen
    \cite{2004R0552_summary}.
    Hierbei definiert die Verordnung ein einheitliches System für die Zertifizierung von Komponenten und Systemen\footnote{darunter \atmans{} Komponenten / -Systeme} sowie den dazugehörigen Verfahren und stellt zudem sicher, dass neue zugelassene und validierte Betriebskonzepte eingeführt werden. \cite[Art.3 Abs.1]{2004R0552}
    Hierbei erklären \ac{ANSP} folglich \acs{EG}-Prüferklärungen (\acs{EGP})\footnote{Gültigkeit bis 2023: Ablösung durch \ac{SoC} unter \acs{EASA} nach \vo{VO}{EU}{1769/2023} \cite{2023R1769}}, welche die Einhaltung aller einschlägigen Verordnungen begründet und der nationalen Aufsichtsbehörde\footnote{\acf{BAF} in Deutschland} vorgelegt werden. \cite[Art. 6 Abs. 1f]{2004R0552}

    \medskip
    Weitere wesentliche Inhalte der Verordnung beinhalten:

    \begin{itemize}
        \item Grundlegende Anforderungen (essential requirements) und
        \item Maßnahmen im Bereich der Interoperabilität;
        \item Gemeinschaftliche Spezifikationen (CS);
        \item Komponenten \ac{EGK} und \ac{EGG}; sowie
        \item verbundene Schutzmaßnahmen und
        \item Übergangsbestimmungen
    \end{itemize}

\paragraph{Verordnung (\acsfont{EU}) Nr. 2018/1139}
    \begin{quote}
        ,,Festlegung gemeinsamer Vorschriften für die Zivilluftfahrt und zur Errichtung einer Agentur der Europäischen Union für Flugsicherheit [...] und zur Aufhebung der Verordnungen (\acs{EG}) \acsfont{Nr.} 552/2004 [...]`` \cite{2018R1139} 
    \end{quote}

    \noindent
    Die Verordnung Nr. 1139\footnote{Nummerierung harmonisiert nach 2015 \cite{eu_number_harmony}} aus dem Jahre 2018 gilt als Überarbeitung der vorhergehenden Verordnung \vo{VO}{EG}{552/2004}, welche neue Vorschriften aller Schlüsselbereiche der Luftfahrt umfasst, um das ,,Wachstum des Luftfahrtsektors in der \ac{EU} zu fördern, den Sektor wettbewerbsfähiger zu machen und Innovationen anzuregen.`` \cite{2018R1139_summary} 

    \medskip
    Im Weiteren enthält die Verordnung ein überarbeitetes Mandat für die Aufgabenbereiche der \acf{EASA}, welche fortan die Aufgaben des \ac{SES}-Pakets I übernimmt, sowie weiteren Vorschriften zu unbemannten Luftfahrzeugen sowie Sport- und Freizeitfliegerei.
    Die unter der \vo{VO}{EG}{552/2004} erlassenen Durchführungsbestimmung behalten im Sinne der Vorbereitung weiterer Rechtsakte vorerst weiter ihre Gültigkeit. \cite[ErwG. 83]{2018R1139}
    
