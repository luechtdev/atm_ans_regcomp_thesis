\subsection{ATM/ANS}

Air Traffic Managament/Air Navigation Services (ATM/ANS), definiert in Verordnung EU 2018/1139 Art. 3 Abs. 5, bezeichnet Flugverkehrsmanagement (ATM) und Flugsicherungsdienste (ANS) und deckt dessen Funktionen und Dienste im Sinne der Rahmenverordnung (siehe \ref{er_549}) ab.
% Darüber hinaus 


Im Kontext der Deutschen Flugsicherung (DFS) dient der Begriff zugleich als Anknüpfungspunkt für die Definition eines Gesamt-Funktionalen Systems einer Organisation (auch \textit{Gesamtsystem Flugsicherung}), wobei jeder zertifizierte Dienst als eigenes sogenanntes \textit{funktionales System} dargestellt wird.
Durch den rechtlichen Rahmen und die verantwortliche Behörde wurde die DFS für die Erbringung folgender Dienste im deutschen Zuständigkeitsbereich zertifiziert: \cite[17]{ba_technik}
\begin{itemize}
    \item   \acf{ATS}, bestehend aus
    \begin{itemize}
        \item \acf{ATC},
        \item Flugalarmdienst,
        \item Fluginformationsdienst;
    \end{itemize}
    \item  Kommunikationsdienst (C);
    \item  Navigationsdienst (N);
    \item  Überwachungsdienst (S);
    \item  \acf{AIS};
    \item  \acf{ATFM};
    \item  \acf{ASM};
    \item  \acf{FPD}
\end{itemize}

\subsection{ATM/ANS Equipment}

ATM/ANS-Equipment (oder auch ATM/ANS-Ausrüstung), definiert in der Durchführungsverordnung EU 2023/1769 Art. 2 Abs. 1, bestimmt Systeme und Komponenten\footnote{jeweils definiert in VO EU 2018/1139 Art. 3 Abs. 6f}, die zur Erbringung der Dienste im funktionalen System beitragen.