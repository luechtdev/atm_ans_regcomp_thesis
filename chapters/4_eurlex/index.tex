\chapter{Datenquelle 1: Europäische Verordnungen und EurLex}

% \begin{quote}
% \textcolor{red}{Einführung in die EU VOs als Datenquelle. Zuerst thematisch in die Arbeitsweise und anschließenden digital, in die Möglichkeiten auf die Daten zuzugreifen.
% Zuerst kurze Einführung, warum EU (naheliegend)}
% \end{quote}



    \section{Europäische Verordnungen}


    Die regulative Grundlage für alle, in dieser Arbeit thematisierten, Anforderungen werden in europäischen Verordnungen -- und jene konsolidierende Medien -- festgehalten.
    Die Arbeitsweise der EU und dessen beteiligte Organe zur Erstellung  wird durch den Vertrag über Europäische Union (EUV) und den Vertrag über die Arbeitsweise der Europäischen Union (AEUV) festgelegt.
    
        
        
        \subsection{Beteiligte Organisationen}


\subsubsection{Europäische Kommission}\footnote{Art. 17 EUV} 

Die Europäische Kommission, mit Sitz in Brüssel, vertritt durch je ein Mitglied alle EU-Staaten (zzt.: 27). Sie verfasst als Kollegium mit einfacher Mehrheit, jedoch werden in der Praxis die meisten Beschlüsse im Konsens gefasst. 

Als einziges Organ verfügt die Kommission über das Initiativrecht im Gesetzgebungsverfahren. Sie kann – unter Umständen auf Aufforderung zum Vorschlag von Rat, Parlament oder Bürgerinitiativen – einen Gesetzesvorschlag annehmen und begleitet ihn im Optimalfall durch den gesamten Gesetzgebungsprozess und ist bestrebt, diesen in enger Zusammenarbeit mit den anderen Organen zu erleichtern.

%%%%%%%%
%%%%%%%%
%%%%%%%%


\subsubsection{Europäisches Parlament}

\begin{center}
    {\footnotesize(Art. 13, 14 Abs. 1 EUV, Art 223f AEUV)}
\end{center}

\noindent
%\lipsum[]

%%%%%%%%
%%%%%%%%
%%%%%%%%

\subsubsection{Rat der Europäischen Union}

\begin{center}
    {\footnotesize(Art. 16 EUV, Art 137ff. AEUV)}
\end{center}

\noindent
Der Rat der Europäischen Union – nicht zu verwechseln mit dem Europäischen Rat – ist ein legislatives Organ der EU mit Sitz in Brüssel welches, bis auf Ausnahmen, zusammen mit dem Europäischem Parlament Rechtsakte beschließt.
Der Rat der EU (auch Ministerrat) setzt sich aus je einem Vertreter auf Ministerebene der Mitgliedsstaaten zusammen. Bundesstaatlich organisierte Staaten können sich jedoch auch durch regionale Regierungsmitglieder vertreten lassen. Hierbei tritt der Rat in ver-schiedenen Fachformationen zusammen:

\begin{itemize}
    \item Allgemeine Angelegenheiten
    \item Wirtschaft- und Finanzen (ECOFIN)
    \item Wettbewerbsfähigkeit
    \item Umwelt
    \item Justiz und Inneres
    \item Landwirtschaft und Fischerei
    \item Verkehr
    \item Telekommunikation und Energie
    \item etc.
\end{itemize}

Im Rat der EU können die Mitgliedstaaten ihre nationalen Interessen mit den zu beschließenden Rechtsakten in Einklang bringen.

Neben den Verordnungen, welche durch das Ordentliche Gesetzgebungsverfahren in Kollaboration mit dem Parlament erlassen werden, kann der Rat ebenfalls Durchführungsvorschriften oder Empfehlungen erlassen.

Neben den Gesetzgebende Aufgaben des Rates stehen ihm auch noch weitere Aufgaben zu. So schließt dieser ebenfalls Internationale Abkommen und Verträge mit Drittstaaten oder internationalen Organisationen. 
Beschlüsse werden einstimmig oder mit qualifizierter Mehrheit beschlossen.


    
            \pagebreak    
        
            
\begin{quote}
\textcolor{green}{Diese Seite auch}
\end{quote}
            \pagebreak    
        \subsection{Ordentliches Gesetzgebungsverfahren}
            \pagebreak    
        
        \subsection{Konsolidierungen von Verordnungen / Lifecycle}
            \pagebreak    
        

    \section{OpenData in der Europäischen Union}

Die Jahre 2003 wurde durch die EU die sogenannte \textit{PSI-Richtlinie} (Re-use of Public Sector Information 2003/98/EG) veröffentlicht, welche einen möglichst einfachen, unbürokratischen und allgemeinen Zugriff auf Informationen ermöglichen soll.
Die Richtlinie wurde im November 2005 auf den Geltungsbereich des EWR ausgeweitet\cite{2005D0105} und im Dezember 2006 durch das \textit{Informationsweiterverwendungsgesetz(IWG)} in nationales Recht umgesetzt.
Auf Basis von erheblichen Änderungen eben dieser Richtlinie im Verlaufe der Zeit entschied sich die Kommission für eine Neufassung der Richtlinie, welche im Jahr 2019 erschien und die alte Richtlinie in seiner Gültigkeit ablöst. \cite[Prä. Abs. 1ff.]{2003L0098}



        
        % \pagebreak
        \subsection{OpenData Project / Vision}

Um die Bemühungen der definierten Initiative für Open-Data weiter zu unterstützen, wurde im Folgenden unter anderem wissenschaftliche Projekte und Arbeiten finanziert.
Beispielsweise 



        
        \pagebreak
        \subsection{EU Cellar Plattform}

        \subsubsection{Motivation}
        
Gestützt durch die PSI-Richtlinie, dem Strategiedokument \textit{Europa 2020} für u.a. der ,,Entwicklung einer auf Wissen und Entwicklung gestützten Wirtschaft`` und dem Arbeitsvertrag der Europäischen Union(AEUV), insbesondere Art. 249, beschließt die EU im Jahre 2011 auch die Weiterverwendung der eigenen Kommissionsdokumente.\cite[Prä. 1]{2011D0833}

        \subsubsection{Umsetzung}

Nach dieser Richtlinie legt sich die Kommission fest, ein Datenportal einzurichten, welches einen zentralen Zugang zu ihren strukturierten Daten ermöglicht und folglich die Verknüpfung und Weiterverwendung für kommerzielle oder nichtkommerzielle Zwecke zu erleichtern. \cite[Art. 5]{2011D0833}
Die hieraus entstandene Cellar Plattform ist ein öffentliches semantisches Repository des \textit{Amts für Veröffentlichung der EU} (OP), welche unter anderem EUR-Lex\footnote{\href{https://eur-lex.europa.eu/homepage.html?locale=de}{https://eur-lex.europa.eu/}}, als digitales Zugangsportal zu europäischem Recht, als auch eine eigene Plattform\footnote{ehemals EU-Bookshop: \href{https://bookshop.europa.eu/}{https://bookshop.europa.eu/}} zur Publikation von weiteren informativen wie wissenschaftlichen Dokumenten ermöglicht.
Die Struktur der Daten basiert auf \textit{Semantic Technologies} und ermöglicht es, Daten auf Basis von definierten Standards weiterzunutzen und zu teilen. 
\cite[5]{eu_cellar}

        \subsubsection{Umsetzung}

Weiter werden Dokumente auch, soweit sinnvoll und nicht mit einem unverhältnismäßigen Mehraufwand verbunden, in einem maschinenlesbaren Format zur Verfügung gestellt \cite[Art. 8 Abs. 1f]{2011D0833}







        % \subsubsection{Unterschiedliche Formate etc}
        \pagebreak
    
    \section{Integration in die Analyse / Prozesse}
    
\begin{quote}
\textcolor{green}{1-2 Seiten}
\end{quote}

\subsection{Übertrag auf das erarbeitete Datenmodell}
\pagebreak