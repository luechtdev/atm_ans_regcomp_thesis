\chapter{Datenquelle 1: Europäische Verordnungen und EurLex}

\begin{quote}
\textcolor{red}{Einführung in die EU VOs als Datenquelle. Zuerst thematisch in die Arbeitsweise und anschließenden digital, in die Möglichkeiten auf die Daten zuzugreifen.
Zuerst kurze Einführung, warum EU (naheliegend)}
\end{quote}

    \section{Europäische Verordnungen}
        \subsection{Beteiligte Organisationen}
    
            \pagebreak    
        
            
\begin{quote}
\textcolor{green}{Diese Seite auch}
\end{quote}
            \pagebreak    
        \subsection{Ordentliches Gesetzgebungsverfahren}
            \pagebreak    
        
        \subsection{Konsolidierungen von Verordnungen / Lifecycle}
            \pagebreak    
        

    \section{OpenData in der Europäischen Union}

Die Jahre 2003 wurde durch die EU die sogenannte \textit{PSI-Richtlinie} (Re-use of Public Sector Information 2003/98/EG) veröffentlicht, welche einen möglichst einfachen, unbürokratischen und allgemeinen Zugriff auf Informationen ermöglichen soll.
Die Richtlinie wurde im November 2005 auf den Geltungsbereich des EWR ausgeweitet\cite{2005D0105} und im Dezember 2006 durch das \textit{Informationsweiterverwendungsgesetz(IWG)} in nationales Recht umgesetzt.
Auf Basis von erheblichen Änderungen eben dieser Richtlinie im Verlaufe der Zeit entschied sich die Kommission für eine Neufassung der Richtlinie, welche im Jahr 2019 erschien und die alte Richtlinie in seiner Gültigkeit ablöst. \cite[Prä. Abs. 1ff.]{2003L0098}



        
        % \pagebreak
        \subsection{OpenData Project / Vision}

Um die Bemühungen der definierten Initiative für Open-Data weiter zu unterstützen, wurde im Folgenden unter anderem wissenschaftliche Projekte und Arbeiten finanziert.
Beispielsweise 



        
        \pagebreak
        \subsection{EU Cellar Plattform}

        \subsubsection{Motivation}
        
Gestützt durch die PSI-Richtlinie, dem Strategiedokument \textit{Europa 2020} für u.a. der ,,Entwicklung einer auf Wissen und Entwicklung gestützten Wirtschaft`` und dem Arbeitsvertrag der Europäischen Union(AEUV), insbesondere Art. 249, beschließt die EU im Jahre 2011 auch die Weiterverwendung der eigenen Kommissionsdokumente.\cite[Prä. 1]{2011D0833}

        \subsubsection{Umsetzung}

Nach dieser Richtlinie legt sich die Kommission fest, ein Datenportal einzurichten, welches einen zentralen Zugang zu ihren strukturierten Daten ermöglicht und folglich die Verknüpfung und Weiterverwendung für kommerzielle oder nichtkommerzielle Zwecke zu erleichtern. \cite[Art. 5]{2011D0833}
Die hieraus entstandene Cellar Plattform ist ein öffentliches semantisches Repository des \textit{Amts für Veröffentlichung der EU} (OP), welche unter anderem EUR-Lex\footnote{\href{https://eur-lex.europa.eu/homepage.html?locale=de}{https://eur-lex.europa.eu/}}, als digitales Zugangsportal zu europäischem Recht, als auch eine eigene Plattform\footnote{ehemals EU-Bookshop: \href{https://bookshop.europa.eu/}{https://bookshop.europa.eu/}} zur Publikation von weiteren informativen wie wissenschaftlichen Dokumenten ermöglicht.
Die Struktur der Daten basiert auf \textit{Semantic Technologies} und ermöglicht es, Daten auf Basis von definierten Standards weiterzunutzen und zu teilen. 
\cite[5]{eu_cellar}

        \subsubsection{Umsetzung}

Weiter werden Dokumente auch, soweit sinnvoll und nicht mit einem unverhältnismäßigen Mehraufwand verbunden, in einem maschinenlesbaren Format zur Verfügung gestellt \cite[Art. 8 Abs. 1f]{2011D0833}







        % \subsubsection{Unterschiedliche Formate etc}
        \pagebreak
    
    \section{Integration in die Analyse / Prozesse}
    
\begin{quote}
\textcolor{green}{1-2 Seiten}
\end{quote}

\subsection{Übertrag auf das erarbeitete Datenmodell}
\pagebreak