% \graffito{Für den Part 2.1 würde ich mir noch ein wenig Feedback / Verbesserungsvorschläge wünschen}

\chapter{Begriffsbestimmungen}

\subsubsection{ATM/ANS}\label{beg:atmans}

Air Traffic Managament/Air Navigation Services (\atmans), definiert in \vo{VO}{EU}{2018/1139} Art. 3 Abs. 5, bezeichnet \acf{ATM} und \acf{ANS} und deckt dessen Funktionen und Dienste im Sinne der Rahmenverordnung (siehe \ref{er_549}) ab.
% Darüber hinaus 
Durch den rechtlichen Rahmen der \vo{DVO}{EU}{2017/373} und der verantwortlichen Behörde wurde die DFS für die Erbringung folgender \atmans{} Dienste im deutschen Zuständigkeitsbereich zertifiziert: \cite[17]{ba_technik}
\begin{itemize}
    \item \acf{ATS}, bestehend aus
    \begin{itemize}
        \item \acf{ATC},
        \item Flugalarmdienst,
        \item Fluginformationsdienst;
    \end{itemize}
    \item  Kommunikationsdienst (C);
    \item  Navigationsdienst (N);
    \item  Überwachungsdienst (S);
    \item  \acf{AIS};
    \item  \acf{ATFM};
    \item  \acf{ASM};
    \item  \acf{FPD}
\end{itemize}

\subsubsection{ATM/ANS Equipment}

ATM/ANS-Equipment (oder auch ATM/ANS-Ausrüstung), definiert in der \acf{DVO} \acs{EU} 2023/1769 Art. 2 Abs. 1, bestimmt Systeme und Komponenten\footnote{jeweils definiert in VO EU 2018/1139 Art. 3 Abs. 6f}, die zur Erbringung der Dienste im funktionalen System beitragen.


\subsubsection{Anforderungen / Durchführungsbestimmungen}



\subsubsection{Funktionales System}

Ein \textit{Funktionales System} (engl. functional system) beschreibt nach der \vo{DVO}{EU}{2017/373} ,,eine Kombination von Verfahren, Personal und Ausrüstung, einschließlich Hardware und Software, zur Erfüllung einer Funktion im Bereich \acs{ATM}/\acs{ANS} und sonstiger Funktionen des Flugverkehrsmanagementnetzes``.
\cite[Anh. I Abs. 56]{2017R0373}
Die \acf{DFS} wird im Rahmen der in \ref{beg:atmans} definierten Dienste auch als Gesamt-Funktionales System (auch \textit{Gesamtsystem Flugsicherung}) qualifiziert.
\cite[17]{ba_technik}

% Im Kontext \vo{DVO}{EU}{2017/373} wird die \acf{DFS} zugleich als Anknüpfungspunkt für die Definition eines Gesamt-Funktionalen Systems einer Organisation (auch \textit{Gesamtsystem Flugsicherung}), wobei jeder zertifizierte Dienst als eigenes sogenanntes \textit{funktionales System} dargestellt wird.

\graffito{Bedürfen noch mehr Begriffe in Bezug auf diesen Abschnitt eine Erläuterung? uU der Begriff "{}Anforderung"}