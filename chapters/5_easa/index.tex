\chapter{Datenquelle 2: EASA und Easy Access Rules}

    \section{European Union Aviation Safety Agency}

\begin{quote}
\textcolor{red}{Einführung zur EASA. Neue Position und Bedeutung für den Prozess. Was sind Easy Access Rules und welche Bedeutung haben sie für uns. Hard law / Soft law. Added benefit durch AMC/GM}
\end{quote}
\begin{quote}
\textcolor{green}{1-2 Seiten}
\end{quote}

        
        \subsection{Bedeutung nach VO 2023/1768ff.}
        
        \pagebreak
        \subsection{Easy Access Rules}

Im Rahmen der behandelten Definitionen aus der Studie zu funktionalen Anforderungen bibliografischer Datensätze \acs{FRBR} (siehe \ref{frbr}) bilden die \textit{Easy Access Rules} eine weitere \textit{Manifestation} des bestehenden \textit{Werkes} der referenzierten, abgedeckten Verordnung.
Auch wenn die \ac{EAR} den Umfang bestehender \textit{Manifestationen} inhaltlich um \textit{Annehmbare Nachweisverfahren} und \textit{Guidance Material} erweitern, so ändern sie nicht 



\begin{quote}
\textcolor{green}{\~1.5 Seiten}
\end{quote}

        \pagebreak
        \subsection{Annehmbare Nachweisverfahren}

        Die \ac{EASA} ist nach der Anforderung \textsf{ATM/ANS.AR.A.015.a} angehalten, annehmbare Nachweisverfahren (engl. Accepted Means of Compliance (AMC)) zu entwickeln, welche zur Bewertung der Compliance herangezogen werden können und dessen Abdeckung eine vollständige Abdeckung der zugrundeliegenden einschlägigen Verordnung garantiert. 
        \cite[Anh. II]{2017R0373}


    \subsubsection{Alternative Nachweisverfahren}
        
        

\subsection{Guidance Material}

\begin{quote}
\textcolor{green}{\~0.5 Seiten}
\end{quote}
        \pagebreak
    \section{EASA eRules Plattform}

    Die \ac{EASA} eRules Plattform ist ,,einheitliche, einfach zugängliche, online Datenbank für alle in der Luftfahrt anwendbaren Regeln für Stakeholder im europäischen Luftraum``.
    \cite[5]{easa_xml_doc}

    Als ein Produkt dieser Plattform werden die sogenannten Easy Access Rules definiert und bis dato (24.6.22) in zwei Formaten veröffentlicht. 

    
Am 26. Januar 2023 veröffentlichte die EASA die Entscheidung, Easy Access Rules fortan auch in einem maschinenlesbaren XML Format Endnutzern bereitzustellen. \cite{easa_xml_publication}


\subsection{Informationsarchitektur}

    Im Rahmen der eRules Plattform und des maschinenlesbaren \ac{XML} Formates, werden Inhalte in kleine sogenannte ,,Topics`` unterteilt.
    Ein Topic repräsentiert eine \ac{EU} Durchführungsbestimmung (siehe \ref{ch:ir}); eine \ac{EU} Delegierte Bestimmung; ein Annehmbares Nachweisverfahren der \ac{EASA}; \ac{EASA} Guidance Material; oder sog. \ac{EASA} \acf{CertS}\footnote{(CS) wegen Überschneidung mit \acf{CS} geändert}. \cite[S. 5f]{easa_xml_doc}
    
Topics werden repräsentativ ihrer konzeptuellen Beziehungen zwischen den unterschiedlichen Topics in der baumartigen Struktur abgebildet.
Hierbei werden Topics, welche Verordnungen oder \acp{CertS} abbilden meist auf einem strukturell höheren Level und alle diesem zugeordneten Topics wie. \acsp{AMC}, \acsp{GM}, oder anderen Gesetzesgrundlagen dargestellt.


        
        \subsubsection{Publikationsformat}

            Die für die Publikation der maschinenlesbaren Dokumente wählte die \ac{EASA} die von \acf{MS} etablierte und standardisierte \textit{Open Packaging Convention (ECMA-376, ISO/IEC 29500-2)}.
            Dieser Standard ist Teil der Familie an \ac{XML} Schemata, welche gemeinsam als \acf{OOXML} bezeichnet werden \textit{(ISO/IEC 29500)} und \ac{XML} Semantik für Textverarbeitungs-, Tabellen\-kal\-ku\-lations- und Präsentationsdokumente -- oder mit diesen konform agierenden Dokumenten -- bereitstellen. 
            \cite[vii]{easa_opc_iso} 
            Dies bedeutet unter anderem, dass valide Dokumente Teil eines größeren Dokumentes sein können, wessen weitere Daten nicht auf die Benutzung als Office-Dokument Einfluss nehmen.
            Genau von dieser Eigenschaft hat die \ac{EASA} Gebrauch gemacht und den Aufbau ihrer digitalen Publikationen in zwei nennenswerte Teile unterteilt.   

            \subsubsection{Anforderungsinhalte}

    Inhalte von Topics werden unter Release 1.0.0 der eRules \ac{XML} Spezifikation als ,,opaque data structure`` angesehen.
    Dies bedeutet, dass im Zug dieses Releases keine inhaltliche Struktur der Topics definiert wurde.
    Es bleibt jedoch den Anwender:innen überlassen, Implementationen auf die Angaben innerhalb des beschriebenen \acs{OOXML} Formates zu stützen.
    \cite[6]{easa_xml_doc}

            Hiernach sind inhaltlichen Anforderungstexte der Easy Access Rules Teil des Textverarbeitungsdokumentes (Part "\textit{/word/document.xml}") und damit auch über eine grafische Textbearbeitungsoberfläche (bspw. \ac{MS} Word) einsehbar und bearbeitbar.
            Hierbei werden Textinhalte des Dokumentes so strukturiert, dass die einzelnen Entitäten voneinander isoliert und über eine -- innerhalb des Dokumentes -- lokal einzigartige ID, eindeutig identifizierbar gemacht werden.
            So können die Inhalte bei Bedarf beliebig bearbeitet werden, ohne die Struktur der Topics oder die Integrität anderer strukturellen Verweise zu gefährden.
            % Außerdem ist es so möglich, Einträge in den Metadaten auf die entsprechenden Anforderungsinhalte verweisen zu lassen. 

            \pagebreak
            \subsubsection{Metadaten}

            Zusätzlich zu den Inhalten ermöglicht es der \ac{OOXML} Standard, weitere Parts an das Textdokument anzuhängen.
            Die \ac{EASA} nutzt diese Möglichkeit, um das Dokument und dessen Anforderungen mit eigenen Metadaten aus dem internen \ac{EASA} \ac{CCMS} zu bereichern.
            Diese Metadaten enthalten interne Informationen zu der Struktur des Dokumentes und den einzelnen Anforderungen und stehen dem Endnutzer in Form eines \ac{OPC} Parts (gemäß 6.2 \cite{easa_opc_iso}) zur Verfügung.
            Dessen Struktur beruht auf einem eigenen \ac{XML} Schema, welches in der entsprechenden Dokumentation der \ac{EASA} definiert ist.

            

\begin{quote}
\textcolor{red}{Analyse der Metadaten}
\end{quote}
        
        % \subsection{Vor- \& Nachteile}
    \pagebreak
    \section{Integration in die Analyse / Prozesse}