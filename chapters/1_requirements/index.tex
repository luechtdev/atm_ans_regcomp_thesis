\chapter{Betriebliche Prozesse zu \atmansf{} Equipment}
    
    \begin{center}
        \footnotesize
        Im Rahmen der DFS Deutsche Flugsicherung GmbH\footnote{Im Rahmen dieser Arbeit wird die Analyse und Modellierung der Prozesse und Daten in Bezug auf den Anwendungsfall der \acs{DFS} betrachtet. Sie bildet das Gegenstück der Analyse der externen Anforderungsdokumente durch \acs{EU} und \ac{EASA}.}
    \end{center}
    
    \noindent
    \atmans{} Equipment stellt die essenzielle Basis aller relevanten Prozesse im Flugsicherungsbetrieb dar und ist für die Abwicklung von Flugbewegungen unabdingbar. 
    Kommunikations-, Navigations- und Überwachungsdienste sowie weitere Systeme zur Unterstützung von Lotsinnen und Lotsen ermöglichen es sicher, effizient und interoperabel Flugsicherung zu betreiben.
    Das Bundesministerium für Digitales und Verkehr verfügt weiter, dass alle diese ,,Flugsicherungsdienste und die dazu erforderlichen flugsicherungstechnischen Einrichtungen`` von denen eine Notwendigkeit im Sinne der \vo{DVO}{EU}{2017/373} Artikel 3(a) ausgeht, entsprechend vorgehalten werden.
    \cite[§27 d]{luftvg}
    
    \medskip
    In seiner Position als deutsche Flugsicherungsorganisation betreibt -- aber auch entwickelt -- die \ac{DFS} Deutsche Flugsicherung GmbH das \atmans{} Equipment für die Erbringung von Flugsicherungsaufgaben im Verantwortungsbereich des deutschen Luftraums 
    \cite[§27 c]{luftvg}.
    Diese, ehemals hoheitliche, Aufgabe wurde nach der Privatisierung der \ac{DFS}\footnote{ehemals Bundesamt für Flugsicherung} 1993 in das \ac{LuftVG} übertragen.
    Seit der Gültigkeit des \ac{SES} Rahmens gilt die \ac{DFS} als zertifizierter deutscher \acf{ANSP} und betreut im Rahmen des funktionalen Luftraumblockes Zentraleuropa (\acs{FABEC}) die Aufgaben \textit{Upper Airspace} und \textit{Approach}. 
    
    \medskip
    Um die verschiedenen Tätigkeiten im Rahmen der konformen Inbetriebhaltung von \atmans{} Equipment zu gewährleisten, definiert die \ac{DFS} einen Rahmen von betrieblichen- und fachlichen Anweisungen zu konzernweiten Rahmenprozessen und internen Anforderungen. \cite[vgl.][]{fa_freigaben,ba_technik}   
    
\section{Produktmanagement}
    
    Das \ac{PdM} einer \atmans{} Ausrüstung ist definitionsgemäß für die Gesamtheit von dessen Führungsaufgaben und Organisation in allen Phasen bis zum Abschluss des Projektes verantwortlich \cite[31]{ba_technik}.
    Hierzu zählt auch das Änderungsmanagement, welches u.a. auf Basis von geänderten Anforderungen entsprechende bundesweite Änderungsverfahren. 
    Das \ac{PdM} wird im Rahmen von Änderungen mit der technischen Umsetzung dieser vom Anforderungsmanagement beauftragt und ist weiter für die Erstellung der technischen Freigabe verantwortlich.

\pagebreak
\section{Anforderungsmanagement}
    
    Das \acf{AM} ist nach Definition der \ac{DFS} für die Ersterstellung sowie die bundesweite Pflege und Weiterentwicklung der Anforderungen an eine (FS-)Technische Einrichtung während ihres Lifecycles zuständig. 
    Hierbei erhebt es betriebliche, betriebstechnische und systemtechnische Anforderungen\footnote{gemeinsam bezeichnet als \textit{angeforderter Sollzustand}} an eine (FS-)Technische Einrichtung bei den Nutzer:innen und vertritt diese gegenüber dem Produkt- und/oder dem Projektmanagement. 
    Das betriebliche Anforderungsmanagement ist für die betriebliche Freigabe bei der Indienststellung sowie der Umrüstung einer FS-Technischen Einrichtung zuständig. \cite[31]{ba_technik}

    \medskip
    Im Rahmen dieses betrieblichen Freigabeprozesses\footnote{bei Indienststellung, Umrüstung oder Änderung der Anforderungen (darunter auch regulative Anforderungen)}, ist es die Aufgabe des Anforderungsmanagements, die Nachweisführung -- gemäß der folgenden Definition -- zu allen einschlägigen externen Anforderungsdokumenten, Verordnungen wie Durchführungsbestimmungen zu erarbeiten.
    
\section{Nachweisführung}
    
    \begin{quote}
        ,,Systeme für das Flugverkehrsmanagement dürfen erst in Dienst gestellt werden, wenn sie auf die Einhaltung der grundlegenden Anforderungen und relevanten \ac{DVO} für die Interoperabilität geprüft wurden und diese einhalten.``
         (\ac{BAF}) \cite[17]{baf_iop}
    \end{quote}
    
    \noindent
    Die Betriebsprozesse der \ac{DFS} fordern, dass es für jede Indienststellung oder Umrüstung einer FS-technischen Einrichtung / Technischen Einrichtung Typ 1\footnote{(FS-)Technische Einrichtung welche die Definition von \atmans{} Equipment erfüllt} eine technische und betriebliche Freigabe zu erstellen gilt 
    \cite{fa_freigaben}.

    \medskip
    Die Verordnungslage der Interoperabilitätsverordnung\footnote{\vo{VO}{EG}{552/2004}} verlangt hierbei, dass vor der Indienststellung eines \textit{Systems} durch die jeweilige Flugsicherungsorganisation eine \acf{EGP} ausgestellt wird, mit der die Einhaltung der Vorschriften erklärt wird, und welche im Anschluss zusammen mit technischen Unterlagen der nationalen Aufsichtsbehörde vorgelegt wird. 
    Die Bestandteile dieser Erklärung und der technischen Unterlagen sind in Anhang IV der selbigen Verordnung aufgeführt. 
    Die nationale Aufsichtsbehörde ist außerdem in der Lage, zusätzliche Informationen anzufordern, die zur Überwachung der Compliance-Erklärung erforderlich sind.
    \cite[Art.6 Abs.2]{2004R0552}

    \medskip
    Das \acf{BAF} -- als nationale Aufsichtsbehörde -- erfüllt in diesem Kontext die Funktion einer ,,funktional unabhängigen``\cite[Art. 4 Abs. 1f]{2004R0549} Behörde, welche nach dem Trennungsgebots §31b Abs. 1 Satz 1 des \ac{LuftVG} i.V.m. Art. 87 d Abs. 1 \ac{GG} gefordert ist.
    \cite[S. 14]{eu_ses_studie} 
     
