\chapter{Betriebliche Prozesse zu ATM/ANS Equipment}

\begin{center}
    \footnotesize
    Im Rahmen der DFS Deutsche Flugsicherung GmbH\footnote{Im Rahmen dieser Arbeit wird die Analyse und Modellierung der Prozesse und Daten in Bezug auf den Anwendungsfall der DFS betrachtet. Sie bietet das Gegenstück der Analyse der externen Anforderungsdokumente durch EU und EASA.}
\end{center}
\noindent

ATM/ANS Equipment stellt die essenzielle Basis aller relevanten Prozesse im Flugsicherungsbetrieb dar und ist für die Abwicklung von Flugbewegungen unabdingbar. 
Kommunikationsdienste, Navigationsdienste, Überwachungsdienste sowie weitere Systeme zur Unterstützung von Lotsen und Lotsinnen ermöglichen es sicher, effizient und interoperabel Flugsicherung zu betreiben.


In seiner Position als deutsche Flugsicherungsorganisation betreibt -- aber auch entwickelt -- die DFS Deutsche Flugsicherung GmbH das ATM/ANS Equipment für die Verwendung in den Flugsicherungsaufgaben im Verantwortungsbereich des deutschen Luftraums. 
Um die verschiedenen Etappen der Entwicklung, Inbetriebnahme, Bereitstellung, Nutzung und Wartung adäquat zu unterstützen, bedarf es 
\cite[§27c]{luftvg}


Die \ac{DFS} definiert im Rahmen von betrieblichen- und fachlichen Anweisungen konzernweiten Rahmenprozesse und Anforderungen an die   


    \section{Betriebliches Anforderungsmanagement}

Das Anforderungsmanagement (AM) ist für die Ersterstellung sowie die bundesweite Pflege und Weiterentwicklung der Anforderungen an eine FS-Technische und/oder Technische Einrichtung während ihres Lifecycle zuständig. Es erhebt betriebliche, betriebstechnische und systemtechnische Anforderungen (angeforderter Sollzustand) an eine FS-Technische und/oder Technische Einrichtung bei den Nutzern und vertritt diese gegenüber dem Produktmanagement und/oder dem Projektmanagement. Das betriebliche Anforderungsmanagement ist für die Betriebliche Freigabe bei der Indienststellung sowie der Umrüstung einer FS-Technischen Einrichtung zuständig. 


    \pagebreak
    \section{Betriebliche Nachweisführung}

Die Betriebsprozesse der \ac{DFS} fordern, dass es für jede Indienststellung oder Umrüstung einer FS-technischen Einrichtung / Technischen Einrichtung Typ 1 eine technische und betriebliche Freigabe zu erstellen gilt. \cite{fa_freigaben}

\subsection{Änderungen funktionaler Systeme}

In Bezug auf Änderungen in der Domäne des ATM/ANS-Equipments definiert die DVO (EU) 2017/373 alle Änderungen an ATM/ANS-Equipment als sogenannte a1-Änderungen (bzw. a1-Changes).



% \begin{quote}
% \textcolor{red}{Evtl. ein wenig doppelt, mit dem davor. Auch wenns nichts das Gleiche ist, kann man es u.U. zusammenfassen. Wie groß soll hier der DFS Bezug sein oder soll es eher universell gehalten werden?}
% \end{quote}

        \pagebreak
        \subsection{Statement of Compliance}

        \begin{quote}
\textcolor{red}{Kurze Definition eines SoC, danach Bedeutung für uns etc.}
\end{quote}

        \pagebreak
        \subsection{EG-Gebrauchtauglichkeitserklärung}
        \subsection{EG-Konformitätserklärung}

\begin{quote}
\textcolor{red}{Definitionen zu beidem als wichtiger Bestandteil der EGP(bis '23). Wie ist da die Bedeutung unter der neuen Regularik?}
\end{quote}
        
        \pagebreak
    \section{Lifecycle von ATM/ANS Equipment}

    \begin{quote}
\textcolor{red}{Allgemeine Informationen aus den BAF Richtlinien zum Lifecycle des ATM/ANS Eq.}
\end{quote}
    
        \subsection{Entity Lifecycle Support Application}

\begin{quote}
\textcolor{red}{Ich weiß nicht, wie wichtig ELSA hier für den Inhalt dieser Arbeit ist. u.U. würde ich es weglassen.}
\end{quote}
    \pagebreak
    \section{Nachweisführung regulativer Anforderungen}

\begin{quote}
\textcolor{red}{Teils DFS Prozesse, wie mit regulativen Anforderungen umgegangen wird, teils auch andere Quellen zu der Thematik (andere Branchen)}
\end{quote}
    
        \subsection{Compliance Matrizen}
        \subsection{Folgen der Nachweisführungsprozesse / Audits}
\begin{quote}
\textcolor{red}{Genauerer Bezug auf die DFS und unseren ATM/ANS Use Case. Grundlage für die Anforderungen für die Impact Analyse}
\end{quote}
        