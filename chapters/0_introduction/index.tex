\chapter{Problemstellung}
    \section{Motivation}

        Die Zuverlässigkeit von ATM/ANS Equipment\footnote{Definition nach EU VO 2023/1769 Art. 2 Abs. 1 \cite{2023R1769}}  -- definiert als alle Geräte und Software, von denen die Interoperabilität des \textit{European air traffic management network} (EATMN) abhängt (folglich ATM/ANS Komponenten\footnote{Definition nach EU VO 2018/1139 Art. 3, Abs. 6 \cite{2018R1139}})  sowie die Zusammenfassung dieser für die ,,Unterstützung für Flugsicherungsdienste in allen Flugphasen`` (folglich ATM/ANS Systeme\footnote{Definition nach EU VO 2018/1139 Art. 3, Abs. 7 \cite{2018R1139}}) -- spielt in der Gewährleistung von Sicherheit und Effizienz im internationalen Luftverkehr eine entscheidende Rolle. 
        Hierbei ist die Einhaltung strikter Regularien zur Funktionalität, Sicherheit und Interoperabilität verschiedener Systeme, vor allem in einem zunehmend einheitlichen europäischen Luftraum, für eine reibungslose, sichere Abwicklung des Luftverkehrs unabdingbar. 
        Im europäischen Luftraum sollen einheitliche rechtlich bindende Rahmenbedingungen dafür Sorge tragen, dass ATM/ANS Equipment einheitlich und sicher betrieben werden können.
        Zur Gewährleistung der einheitlichen Anwendung und Einhaltung wurden hierfür durch die Kommission Vorschriften definiert, welche die Verfahren der Nachweisführung zu ATM/ANS Equipment sowie die Rechte und Pflichten dessen Inhaber beschreiben. \cite[Art. 43 Abs. 1]{2018R1139} 
        
    
        Die korrekte Abbildung von rechtlichen Konsolidierungen im Rahmen der Auswirkungen auf die Compliance der ATM/ANS Komponenten durch interne Analysen erfordert aufgrund des großen Umfangs und unterschiedlicher Lifecycle vieler Anforderungsdokumente einen hohen manuellen Aufwand.
        Durch eine automatisierte regulatorische Impact Analysis(RIA) sollen Betreiber und Entwickler von ATM/ANS Equipment besser abschätzen können, in welchem Umfang ihre Komponenten von Änderungen betroffen sind. 


    Auch das Interesse an automatischer Datenverarbeitung von Rechtsdokumenten ist in den letzten Jahren stark gewachsen.
    Unternehmen profitieren fortgehend von immer neuen Werkzeugen und Prozessen, die es ermöglichen rechtliche Informationen einzulesen, zu verarbeiten oder zu verknüpfen.
    Jedoch beschränken sich diese bestehenden Lösungen teils auf einzelne Anwendungsfälle, lokale Zuständigkeitsbereiche oder stellen ihre Daten nur intransparent zur Verfügung\cite[385]{eu_open_legal_info}; auch wenn diese Informationen essenziell für rechtliche Datenanalysen -- oder Impact Analysen -- sind.
    
        
    \pagebreak
    \section{Definition und Rahmen}

        In dem Rahmen dieser Arbeit soll eine grundsätzliche Einführung in die, mit der Entwicklung von ATM/ANS Equipment verbundenen Nachweisführungsprozessen, deren regulativen Basisverordnungen sowie der Beteiligung und Funktion externer Stakeholder darlegen.
        Für die Evaluation von Betriebsprozessen bezieht sich diese Arbeit auf das Betriebsumfeld der \textit{DFS - Deutsche Flugsicherung GmbH} (DFS) in seiner Position als ANSP und in seiner Funktion sowohl als Hersteller, Inbetriebhalter als auch als Nutzer von ATM/ANS Equipment.

        Fortan soll geprüft werden, welche Anforderungen erforderlich sind, um auf Basis der Anforderungsdokumente -- in digitaler Form -- die Auswirkungen von Änderungen auf interne Anforderungsprozesse und Complianceangaben abzubilden.

    \dots



        Hierfür analysiert diese Arbeit unter anderem zwei öffentliche Datenquellen, um dessen Integration für eine mögliche Impactanalyse zu evaluieren.
        \begin{itemize}
            \item Daten von der EU (Publications Office): Cellar API / FMX4 Format
            \item Daten von der EASA: Easy Access Ruless
        \end{itemize}

        Beide Institutionen stellen hierbei Nutzern Daten in Bezug auf regulatorische Anforderungen für ATM/ANS Equipment zur Verfügung.
        Zwischen der Nutzung dieser Daten sowie der rechtlichen Bedeutung und Wirksamkeit deren Anforderungen gilt es zwischen den Quellen zu unterscheiden.
        Dieser, sowie weitere Unterschiede der analysierten Daten/-formate, vor allem in Aspekten bzgl. Vergleichbarkeit, Zugang, Lifecycle und Gültigkeit, sollen im Folgenden verglichen werden.

    
    % \section{Zeitliche Einordnung}

\begin{center}
    \textsc{Disclaimer: \\
    Zeitliche Einordnung der Arbeit}
\end{center}

Diese Arbeit entsteht in der Zeit von Dezember 2023 bis März 2024.
Auch nach Inkrafttreten der Delegierten Verordnungen (EU) 2023/1768ff. im September 2023 ist die Adoption der darin beschriebenen Prozesse zur Zeit des Verfassens noch nicht vollständig umgesetzt. 
Das \ac{BAF} stellt zwischenzeitlich eigene Anforderungen zu der Nachweisführung, welche die Zeit der europäischen Adoption und der Übernahme der \ac{EASA} sowie dessen Spezifikation der Prozesse, überbrücken soll.
Aufgrund der noch fehlenden Spezifikationen zur Nachweisführung der \ac{EASA} sowie den entsprechenden Businessprozessen der Flugsicherungen bezieht sich diese Arbeit in Teilen der Analyse und Definition der Prozesse noch auf Prozesse zur Nachweisführung unter dem \ac{BAF}. 



    % \begin{quote}
    %     \textcolor{red}{Inhaltlich: Evtl. wie die DFS diese Prozesse momentan händisch bearbeitet -> Motivation. Evtl. doppelt es sich auch an anderer Stelle. Vielleicht eher "Betriebliches Interesse" o.ä. }
    % \end{quote}
